\documentclass[10pt, letterpaper]{article}

% Packages:
\usepackage[
    ignoreheadfoot, % set margins without considering header and footer
    top=2 cm, % seperation between body and page edge from the top
    bottom=2 cm, % seperation between body and page edge from the bottom
    left=2 cm, % seperation between body and page edge from the left
    right=2 cm, % seperation between body and page edge from the right
    footskip=1.0 cm, % seperation between body and footer
    % showframe % for debugging 
]{geometry} % for adjusting page geometry
\usepackage{titlesec} % for customizing section titles
\usepackage{tabularx} % for making tables with fixed width columns
\usepackage{array} % tabularx requires this
\usepackage[dvipsnames]{xcolor} % for coloring text
\definecolor{primaryColor}{RGB}{0, 0, 0} % define primary color
\usepackage{enumitem} % for customizing lists
\usepackage{fontawesome5} % for using icons
\usepackage{amsmath} % for math
\usepackage[
    pdftitle={Christopher Odom's Resume},
    pdfauthor={Christopher Odom},
    pdfcreator={LaTeX with RenderCV},
    colorlinks=true,
    urlcolor=primaryColor
]{hyperref} % for links, metadata and bookmarks
\usepackage[pscoord]{eso-pic} % for floating text on the page
\usepackage{calc} % for calculating lengths
\usepackage{bookmark} % for bookmarks
\usepackage{lastpage} % for getting the total number of pages
\usepackage{changepage} % for one column entries (adjustwidth environment)
\usepackage{paracol} % for two and three column entries
\usepackage{ifthen} % for conditional statements
\usepackage{needspace} % for avoiding page brake right after the section title
\usepackage{iftex} % check if engine is pdflatex, xetex or luatex

% Ensure that generate pdf is machine readable/ATS parsable:
\ifPDFTeX
    \input{glyphtounicode}
    \pdfgentounicode=1
    \usepackage[T1]{fontenc}
    \usepackage[utf8]{inputenc}
    \usepackage{lmodern}
\fi

\usepackage{charter}

% Some settings:
\raggedright
\AtBeginEnvironment{adjustwidth}{\partopsep0pt} % remove space before adjustwidth environment
\pagestyle{empty} % no header or footer
\setcounter{secnumdepth}{0} % no section numbering
\setlength{\parindent}{0pt} % no indentation
\setlength{\topskip}{0pt} % no top skip
\setlength{\columnsep}{0.15cm} % set column seperation
\pagenumbering{gobble} % no page numbering

\titleformat{\section}{\needspace{4\baselineskip}\bfseries\large}{}{0pt}{}[\vspace{1pt}\titlerule]

\titlespacing{\section}{
    % left space:
    -1pt
}{
    % top space:
    0.3 cm
}{
    % bottom space:
    0.2 cm
} % section title spacing

\renewcommand\labelitemi{$\vcenter{\hbox{\small$\bullet$}}$} % custom bullet points
\newenvironment{highlights}{
    \begin{itemize}[
        topsep=0.10 cm,
        parsep=0.10 cm,
        partopsep=0pt,
        itemsep=0pt,
        leftmargin=0 cm + 10pt
    ]
}{
    \end{itemize}
} % new environment for highlights


\newenvironment{highlightsforbulletentries}{
    \begin{itemize}[
        topsep=0.10 cm,
        parsep=0.10 cm,
        partopsep=0pt,
        itemsep=0pt,
        leftmargin=10pt
    ]
}{
    \end{itemize}
} % new environment for highlights for bullet entries

\newenvironment{onecolentry}{
    \begin{adjustwidth}{
        0 cm + 0.00001 cm
    }{
        0 cm + 0.00001 cm
    }
}{
    \end{adjustwidth}
} % new environment for one column entries

\newenvironment{twocolentry}[2][]{
    \onecolentry
    \def\secondColumn{#2}
    \setcolumnwidth{\fill, 6.0 cm}
    \begin{paracol}{2}
}{
    \switchcolumn \raggedleft \secondColumn
    \end{paracol}
    \endonecolentry
} % new environment for two column entries

\newenvironment{threecolentry}[3][]{
    \onecolentry
    \def\thirdColumn{#3}
    \setcolumnwidth{, \fill, 4.5 cm}
    \begin{paracol}{3}
    {\raggedright #2} \switchcolumn
}{
    \switchcolumn \raggedleft \thirdColumn
    \end{paracol}
    \endonecolentry
} % new environment for three column entries

\newenvironment{header}{
    \setlength{\topsep}{0pt}\par\kern\topsep\centering\linespread{1.5}
}{
    \par\kern\topsep
} % new environment for the header

\newcommand{\placelastupdatedtext}{% \placetextbox{<horizontal pos>}{<vertical pos>}{<stuff>}
  \AddToShipoutPictureFG*{% Add <stuff> to current page foreground
    \put(
        \LenToUnit{\paperwidth-2 cm-0 cm+0.05cm},
        \LenToUnit{\paperheight-1.0 cm}
    ){\vtop{{\null}\makebox[0pt][c]{
        \small\color{gray}\textit{Last updated in July 2024}\hspace{\widthof{Last updated in July 2024}}
    }}}%
  }%
}%

% save the original href command in a new command:
\let\hrefWithoutArrow\href

% new command for external links:


\begin{document}
    \newcommand{\AND}{\unskip
        \cleaders\copy\ANDbox\hskip\wd\ANDbox
        \ignorespaces
    }
    \newsavebox\ANDbox
    \sbox\ANDbox{$|$}

    \begin{header}
        \fontsize{25 pt}{25 pt}\selectfont Christopher Odom

        \vspace{5 pt}

        \normalsize
        \mbox{Greater Boston, MA}%
        \kern 5.0 pt%
        \AND%
        \kern 5.0 pt%
        \mbox{\hrefWithoutArrow{mailto:christopher.r.odom@gmail.com}{christopher.r.odom@gmail.com}}%
        \kern 5.0 pt%
        \AND%
        \kern 5.0 pt%
        \mbox{\hrefWithoutArrow{tel:+781-475-3804}{781 475 3804}}%
        \kern 5.0 pt%
        \AND%
        \kern 5.0 pt%
        \mbox{\hrefWithoutArrow{https://codom.github.io/}{codom.github.io}}%
        \kern 5.0 pt%
        \AND%
        \kern 5.0 pt%
        \mbox{\hrefWithoutArrow{https://linkedin.com/in/christopher-r-odom}{linkedin.com/in/christopher-r-odom}}%
        \kern 5.0 pt%
        \AND%
        \kern 5.0 pt%
        \mbox{\hrefWithoutArrow{https://github.com/codom}{github.com/codom}}%
    \end{header}

    \vspace{5 pt - 0.3 cm}


%    \section{Welcome To RenderCV!}
%
%
%
%        
%        \begin{onecolentry}
%            \href{https://github.com/sinaatalay/rendercv}{RenderCV} is a LaTeX-based CV/resume framework. It allows you to create a high-quality CV or resume as a PDF file from a YAML file, with \textbf{full Markdown syntax support} and \textbf{complete control over the LaTeX code}.
%        \end{onecolentry}
%
%        \vspace{0.2 cm}
%
%        \begin{onecolentry}
%            The boilerplate content is taken from \href{https://github.com/dnl-blkv/mcdowell-cv}{here}, where a \textit{clean and tidy CV} pattern is proposed by \textbf{\href{https://www.gayle.com/}{Gayle Laakmann McDowell}}.
%        \end{onecolentry}
%
%
%    
%    \section{Quick Guide}
%
%    \begin{onecolentry}
%        \begin{highlightsforbulletentries}
%
%
%        \item Each section title is arbitrary, and each section contains a list of entries.
%
%        \item There are 7 unique entry types: \textit{BulletEntry}, \textit{TextEntry}, \textit{EducationEntry}, \textit{ExperienceEntry}, \textit{NormalEntry}, \textit{PublicationEntry}, and \textit{OneLineEntry}.
%
%        \item Select a section title, pick an entry type, and start writing your section!
%
%        \item \href{https://docs.rendercv.com/user_guide/}{Here}, you can find a comprehensive user guide for RenderCV.
%
%
%        \end{highlightsforbulletentries}
%    \end{onecolentry}
    
    \section{Experience}



        
        \begin{twocolentry}{
            November 2023 – Present
        }
            \textbf{Data Labeler for Software} Data Annotation, Remote -- US\end{twocolentry}

        \vspace{0.10 cm}
        \begin{onecolentry}
            \begin{highlights}
                \item Labelled datasets for coding and data-science workflows
                \item Created various workflows to optimize deploying projects in React, Android, Python, and C++
                \item Leveraged AI to build various file-hosting, music streaming, and blogging apps
            \end{highlights}
        \end{onecolentry}


        \vspace{0.2 cm}

        \begin{twocolentry}{
            June 2020 – August 2020
        }
            \textbf{Software Engineering Intern}, Red Hat -- Westford, MA\end{twocolentry}

        \vspace{0.10 cm}
        \begin{onecolentry}
            \begin{highlights}
                \item Implemented product improvements to Ceph Dashboard API
            \end{highlights}
        \end{onecolentry}


        \vspace{0.2 cm}

        \begin{twocolentry}{
            June 2018 – Jan 2019
        }
            \textbf{Software Engineering Intern}, Draper -- Cambridge, MA\end{twocolentry}

        \vspace{0.10 cm}
        \begin{onecolentry}
            \begin{highlights}
              \item Designed and implemented a test-application framework for hardware based compliance testing on a CRS2 platform.
              \item Learned to build custom ioctls to integrate special hardware tests to an offline data collection pipeline.
              \item Created custom internal tooling to collect data and configure specialized tests.
            \end{highlights}
        \end{onecolentry}

    
    \section{Projects}

        \begin{twocolentry}{
            2023 - 2024 (Unpublished)
        }
            \textbf{2D Game Engine}\end{twocolentry}

        \vspace{0.10 cm}
        \begin{onecolentry}
            \begin{highlights}
                \item Developing a 2D game engine built on top of Raylib that leverages several technologies to help create compelling narratives
                \item Interactive event-based asset management using inotify on linux
                \item Custom async requests library using a curl worker thread
                \item Custom script language and data format for dialogue systems
                \item Tools Used: Zig, C, Raylib, Curl, Gitea
            \end{highlights}
        \end{onecolentry}


        \vspace{0.2 cm}

        \begin{twocolentry}{
            \href{https://github.com/codom/codom.github.io}{github.com/codom/codom.github.io}
        }
        \textbf{Personal Website}\end{twocolentry}

        \vspace{0.10 cm}
        \begin{onecolentry}
            \begin{highlights}
              \item Simple landing page I use to experiment with web development
              \item Historically built using a makefile, now I use vue.js and related build tools
              \item Tools used: vue.js, three.js, glsl, Python, Github Actions
            \end{highlights}
        \end{onecolentry}


        \vspace{0.2 cm}

        \begin{twocolentry}{
            \href{https://github.com/Codom/SimpleGuitarAmp}{github.com/Codom/SimpleGuitarAmp}
        }
            \textbf{Guitar Amp Sim}\end{twocolentry}

        \vspace{0.10 cm}
        \begin{onecolentry}
            \begin{highlights}
                \item Implemented filter algorithms in order to simulate a guitar amp
                \item Packaged to support any DAW with CLAP support (ie, Bitwig)
                \item Tools Used: Zig, C
            \end{highlights}
        \end{onecolentry}

    \section{Additional Experience And Awards}

        \begin{onecolentry}
            \textbf{Caretaker (2021-2024):} Assisted family for end-of-life planning.
        \end{onecolentry}

        \vspace{0.2 cm}

        \begin{onecolentry}
            \textbf{CCDC (2020):} Participated in a competitive regional cybersecurity competition
        \end{onecolentry}

    \section{Technologies}
        \begin{onecolentry}
            \textbf{Languages:} C++, C, Java, SQL, JavaScript, Python, Zig
        \end{onecolentry}

        \vspace{0.2 cm}

        \begin{onecolentry}
            \textbf{Software:} Git, Github, Linux, Docker, Podman, LLVM
        \end{onecolentry}

    \section{Education}
        \begin{twocolentry}{
            Sept 2017 – May 2021 
        }
            \textbf{University of Massachusetts, Lowell}, BS in Computer Science\end{twocolentry}

        \vspace{0.10 cm}
        \begin{onecolentry}
            \begin{highlights}
                \item \textbf{Coursework:} Computer Architecture, Operating Systems, Analysis of Algorithms, Foundations of CS
            \end{highlights}
        \end{onecolentry}

\end{document}
